\documentclass[12pt]{article}
\usepackage{float, amsmath, amssymb, amsthm, algorithm, algorithmic, graphicx, caption, subcaption, mathrsfs, color, cancel, verbatim, cite, authblk, mathtools}
\usepackage{enumitem}

\def\upint{\mathchoice%
    {\mkern13mu\overline{\vphantom{\intop}\mkern7mu}\mkern-20mu}%
    {\mkern7mu\overline{\vphantom{\intop}\mkern7mu}\mkern-14mu}%
    {\mkern7mu\overline{\vphantom{\intop}\mkern7mu}\mkern-14mu}%
    {\mkern7mu\overline{\vphantom{\intop}\mkern7mu}\mkern-14mu}%
  \int}
\def\lowint{\mkern3mu\underline{\vphantom{\intop}\mkern7mu}\mkern-10mu\int}

\let\oldemptyset\emptyset
\let\emptyset\varnothing

\setlength{\oddsidemargin}{0in}
\setlength{\evensidemargin}{\oddsidemargin}
\setlength{\textwidth}{6.5in}
\setlength{\topmargin}{-.25in}
\setlength{\headheight}{0in}
\setlength{\headsep}{0in}
\setlength{\topskip}{0in}
\setlength{\textheight}{9.5in}
\font\bigbf = cmbx10 scaled \magstep1
\font\medbf = cmbx10 scaled \magstephalf
\font\medrm = cmr10 scaled \magstephalf
\font\bigrm = cmr10 scaled \magstep1

\usepackage[english]{babel}
\usepackage[utf8]{inputenc}
\usepackage[colorinlistoftodos]{todonotes}

\title{Harmonic Analysis}
\begin{document}
\noindent \textbf{Lecture 7: Eyvindur Palsson} \\
\noindent www.math.vt.edu/people/palsson/pcmi.html \\

\noindent \textbf{Fourier Series} \\
\noindent If $f$ is an integrable function on a given interval $[a,b]$ of length $L$, then the $n$th Fourier coefficient of $f$ is defined by 
$$ \hat{f}(n) = \frac{1}{L}\int^b_a e^{-2\pi inx/L}dx, n \in \mathbb{Z}$$
and its Fourier series is given by 
$$\sum^\infty_{n=-\infty} \hat{f}(n) e^{2\pi inx/L}$$

\noindent \textbf{Important Question}: When does $\sum^\infty_{n=-\infty} \hat{f}(n) e^{2\pi inx/L}$ converge to $f(x)$?

\noindent If we think about it in an applied setting, basically, we take the encoding as it is, and try to recover the original signal. \\

\noindent \textbf{Example}: $f(x)=x$ on $[0,2\pi]$, 
$$\hat{f}(n) = \frac{1}{2\pi} \int^{2\pi}_0 x e^{-inx} dx = \text{ integration by parts ensues } = \frac{i}{n}$$
when $n \not = 0$. 
$$\hat{f}(0) = \frac{1}{2\pi} \int^{2 \pi}_0 x dx = \pi$$

\noindent The Fourier Series \\
\begin{align*}
\pi + \int^\infty_{n=-\infty} \frac{i}{n} e^{inx} &= \pi - 2 (\frac{1}{1} \sin(x) + \frac{1}{2} \sin (2x) + \cdots) \\
&\uparrow \\
& \frac{i}{n}e^{inx} + \frac{i}{-n} e^{-inx} = -\frac{2}{n} \sin(nx)
\end{align*}

\noindent This yields $\pi$ if $x=0$ or $x=2\pi$, while $f(0)=0$ and $f(2\pi)=2\pi$. Is this an issue?

(1) \\

\noindent \textbf{Domains}:
$$[a,b] \hspace{25pt} \mathbb{R} \hspace{25pt} f(x) = F(e^ix), x \in [0,2\pi]$$

\noindent \textbf{Uniqueness} \\
\noindent \textbf{Theorem}: Suppose that $f$ is integrable, and bounded on an interval with $\hat{f}(n)=0$ for all $n \in \mathbb{Z}$. Then $f(x_0)=0$ whenever $f$ is continuous at $x_0$. \\

\noindent \textbf{Partial Sums}: $$S_N(f)(x):= \sum^N_{n=-N}\hat{f}(n) e^{2\pi inx/L} \text{ on } [0,L]$$

\textbf{Question}: In what sense does $S_Nf \rightarrow f$. \\

\noindent \textbf{Theorem}: Suppose that $f$ is continuous on the circle and the Fourier series of $f$ is absolutely convergent
$$\sum^\infty_{n=-\infty} \vert \hat{f}(n)\vert < \infty $$
Then, 
$$\lim_{N\rightarrow \infty} S_N(f)(x)=f(x) \text{ uniformly in } x$$


\begin{proof}
Since $\sum^\infty_{n=-\infty} \vert \hat{f}(n)\vert < \infty $, then $\sum^\infty_{n=-\infty} \hat{f}(n)e^{inx}$ converges absolutely and in fact uniformly. If $g(x) = \sum^{\infty}_{n=-\infty} \hat{f}(n) e^{inx}$, then $g$ must be continuous. How much different could $g$ be from $f$? Since $f-g$ is continuous and $\hat{f-g}(n)=0$ for all $n \in \mathbb{Z}$, we conclude by uniqueness theorem that $f=g$.  
\end{proof}

\noindent \textbf{Note}: 
\begin{align*}\hat{g}(n) &= \frac{1}{2\pi} \int^{2\pi}_0 g(x) e^{-inx} dx \\
&= \frac{1}{2\pi} \int^{2\pi}_0 \Big(\lim_{N\rightarrow \infty} \sum^N_{n=-\infty} \hat{f}(k)e^{ikx}\Big)e^{-inx}dx  \\
&= \lim_{N\rightarrow \infty} \sum^N_{n=-\infty} \hat{f}(k) \frac{1}{2\pi} \int^{2\pi}_0 e^{i(k-n)x}dx, 0 \text{ if } k=n \text{ and } 0 \text{ otherwise} \\
&= \hat{f}(n)
\end{align*}

\noindent \textbf{Theorem}:  Let $f \in C^2[0,2\pi]$ and $2\pi$ period. Then $\hat{f}(n) = O(\frac{1}{n^2})$ as $\vert n \vert \rightarrow \infty$ and thus the Fourier series of $f$ converges absolutely and uniformly to $f$. \\

\noindent Note: $f(x) = O(g(x))$ as $x \rightarrow a$ means that there exists a constant $C$ such that $\vert f(x) \vert \leq C \vert g(x)\vert$ as $x \rightarrow a$. 

\begin{proof}
Since $f''$ is continuous on $[0,2\pi]$, it is bounded, say $\vert f''(x) \vert \leq B$ for all $x \in (0, 2\pi]$. 
$$\hat{f}(n) = \frac{1}{2\pi} \int^{2\pi}_0 f(x)e^{-inx} dx = \text{ integration by parts ensues } = \frac{1}{in} \hat{f'}(n)$$

\noindent By iterating, we obtain $\hat{f}(n) = \frac{1}{i^2n^2} \hat{f''}(n)$. So,
$$\vert \hat{f}(n) \vert = \frac{1}{n^2} \vert \frac{1}{2\pi} \int^{2\pi}_0 f''(x)e^{-inx} dx \vert  \leq \frac{1}{n^2} \frac{1}{2\pi} \int^{2\pi}_0 B dx = \frac{B}{n^2}$$
\end{proof}

\noindent \textbf{Corollary}: $\hat{f'}(n) = in \hat{f}(n)$. 

\noindent \textbf{Convolution}: Given $2\pi$ periodic, integrable functions $f$ and $g$ on $\mathbb{R}$ define their convolution $f * g$ by $$(f*g)(x) = \frac{1}{2\pi} \int_{-\pi}^{\pi}f(y)g(x-y)dy,$$ for $x\in[-\pi, \pi]$.
This operation has many good properties, but importantly
\begin{enumerate}[itemsep=0pt, parsep=0pt, partopsep=0pt, topsep=0pt]
\item $f*g =g*f$ 
\item $\widehat{f*g}(n) = \hat{f}(n)\hat{g}(n)$. 
\end{enumerate}
\end{document}