\documentclass[12pt]{article}
\usepackage{float, amsmath, amssymb, amsthm, algorithm, algorithmic, graphicx, caption, subcaption, mathrsfs, color, cancel, verbatim, cite, authblk, mathtools}
\usepackage{enumitem}

\def\upint{\mathchoice%
    {\mkern13mu\overline{\vphantom{\intop}\mkern7mu}\mkern-20mu}%
    {\mkern7mu\overline{\vphantom{\intop}\mkern7mu}\mkern-14mu}%
    {\mkern7mu\overline{\vphantom{\intop}\mkern7mu}\mkern-14mu}%
    {\mkern7mu\overline{\vphantom{\intop}\mkern7mu}\mkern-14mu}%
  \int}
\def\lowint{\mkern3mu\underline{\vphantom{\intop}\mkern7mu}\mkern-10mu\int}

\let\oldemptyset\emptyset
\let\emptyset\varnothing

\setlength{\oddsidemargin}{0in}
\setlength{\evensidemargin}{\oddsidemargin}
\setlength{\textwidth}{6.5in}
\setlength{\topmargin}{-.25in}
\setlength{\headheight}{0in}
\setlength{\headsep}{0in}
\setlength{\topskip}{0in}
\setlength{\textheight}{9.5in}
\font\bigbf = cmbx10 scaled \magstep1
\font\medbf = cmbx10 scaled \magstephalf
\font\medrm = cmr10 scaled \magstephalf
\font\bigrm = cmr10 scaled \magstep1

\usepackage[english]{babel}
\usepackage[utf8]{inputenc}
\usepackage[colorinlistoftodos]{todonotes}

\title{Harmonic Analysis}
\begin{document}
\noindent \textbf{Lecture 9: Eyvindur Palsson} \\
\noindent www.math.vt.edu/people/palsson/pcmi.html \\

\noindent \textbf{Convolution} \\
\noindent Recall the convolution operation between two functions $f$ and $g$, 
$$(f * g)(x) = \frac{1}{2\pi} \int^\pi_{-\pi} f(y)g(x-y)\, dy$$ \\

\noindent \textbf{Dirichlet Kernel} \\
We obtain the Dirichlet Kernel by looking at the partial sums for our Fourier series. Let $S_N(f)(x)$ be the $N$th partial sum of the function $f$:

\begin{align*}
S_N(f)(x) &= \sum^N_{n=-N} \hat{f}(n)e^{inx} \\
&= \sum^N_{n=-N} \Big(\frac{1}{2\pi} \int^{\pi}_{-\pi} f(y) e^{-iny} \,dy\Big) e^{inx} && \text{ by definition of } \hat{f}(n) \\
&= \frac{1}{2\pi} \int^{\pi}_{-\pi} f(y) \sum^N_{n=-N} e^{in(x-y)} \, dy \\
&= \frac{1}{2\pi} \int^{\pi}_{-\pi} f(y) D_N(x-y) \, dy &&\text{where } D_N(x-y)= \sum^N_{n=-N} e^{in(x-y)} \\
&= (f * D_N)(x) && \text{ by convolution}
\end{align*} 
We call the function $D_N(x)$ the $N$th Dirichlet Kernel. There are a few important facts to be aware of about the Dirichlet Kernel:
\begin{enumerate}[itemsep=0pt, topsep=0pt, parsep=0pt, partopsep=0pt] 
\item As with the Poisson Kernel, we have the analogous property, 
$$ \frac{1}{2\pi} \int^{\pi}_{-\pi} D_n(x) \, dx = 1$$
\item We have a closed form for the Dirichlet Kernel, which is given by,
$$D_N(x) = \frac{\sin ((N+\frac{1}{2})x)}{\sin(\frac{x}{2})}.$$
\item And, we have an upper bound in the size of the Dirichlet Kernel given by the following
$$\vert D_N(x) \vert \leq C \cdot \min \Big\{ N, \frac{1}{\vert x \vert}\Big\} \text{ if } x\in [-\pi, \pi] \text{ and } N\geq 1$$
\end{enumerate}

\noindent \textbf{Good Kernel on the Circle} \\
\noindent A family of kernels $\{K_n(x)\}^\infty_{n=1}$ on the circle are said to be a family of good kernels if
\begin{enumerate}[itemsep=0pt, topsep=0pt, parsep=0pt, partopsep=0pt] 
\item For all $n\geq 1$,
$$\frac{1}{2\pi} \int^{\pi}_{-\pi} K_n(x) \, dx = 1$$
\item There exists $M>0$ such that 
$$\int^\pi_{-\pi} \vert K_n(x) \vert \, dx \leq M$$
\item For every $\delta>0$,
$$\int_{\delta \leq \vert x \vert \leq \pi} \vert K_n(x) \vert \, dx \rightarrow 0 \text{ as } n \rightarrow \infty$$
\end{enumerate}
\noindent For kernels, we want larger coefficients closer to $n=0$ and smaller coefficients as $\vert n \vert$ gets large. This allows for good approximation without needing to compute the coefficients up to a large value of $n$ (whether negative or positive). A benefit of good kernels is given in the following theorem: \\

\noindent \textbf{Theorem}: Let $\{K_n\}$ be a family of good kernels and let $f$ be bounded and integrable on the circle. Then,
$$ \lim_{n\rightarrow \infty} (f * K_n)(x)=f(x)$$
whenever $f$ is continuous at $x$. If $f$ is continuous everywhere, then the convergence is uniform. \\

\noindent The question is now, is the Dirichlet kernel a good kernel? Unfortunately not, this is seen if we analyze the second condition,
$$\int^\pi_{-\pi} \vert D_n(x) \vert \, dx \geq C \cdot \log(N) \text{ as } N \rightarrow \infty.$$
\noindent Since $D_n$ is bounded below by $C \cdot \log(N)$ for some constant $C$, it is clear that the value tends towards infinity as $N$ gets large (although slow it is still a problem). Therefore, we can conclude that $\{D_n\}$ is not a family of good kernels. So, why did we even bring up Dirichlet kernels? Well, it turns out if we weaken the notion of convergence, we can obtain a good kernel called the Fej\'er Kernel. \\

\noindent \textbf{Fej\'er Kernel} \\
\noindent We begin with defining a new type of convergence for series. Let $S=\sum^\infty_{k=0} c_k$ be a series, we say $S$ is Ces\'aro summable to $\sigma$ if $\lim_{N\rightarrow \infty} \sigma_N = \sigma$, where $\sigma_N = \frac{S_0 + S_1 + \cdots + S_{N-1}}{N}$ is the $N$th Ces\'aro mean. So, how is this notion of convergence for series weaker? We give an example. \\

\noindent \textbf{Example}: Recall that the series $\sum^\infty_{k=0}(-1)^k$ does not converge since the sum oscillates between 1 and 0; however, this series is Ces\'aro summable to $\frac{1}{2}$. \\

\noindent For Fourier series, the $N$th Ces\'aro mean is
\begin{align*}
\sigma_N(f)(x) &= \frac{S_0(f)(x)+ \cdots + S_{N-1}(f)(x)}{N} \\
&= \frac{(f*D_0)(x) + \cdots + (f*D_{N-1})(x)}{N} \\
&= \Big(f*\Big(\frac{D_0+ \cdots + D_{N-1}}{N}\Big)\Big)(x) \\
&= (f*F_n)(x) && \text{where } F_N =\frac{D_0 + \cdots + D_{N-1}}{N} 
\end{align*}
\noindent We call $F_N$ the Fej\'er Kernel. We list some facts about the Fej\'er Kernel. \\

\begin{enumerate}[itemsep=0pt, topsep=0pt, parsep=0pt, partopsep=0pt] 
\item For all $n\geq 1$,
$$\frac{1}{2\pi} \int^{\pi}_{-\pi} F_n(x) \, dx = 1$$
\item We have a closed form for the Fej\'er Kernel, which is given by,
$$F_N(x) = \frac{1}{N} \frac{\sin^2(\frac{Nx}{2})}{\sin^2(\frac{x}{2})}$$
\item $\{F_N\}$ is a family of good kernels (the second condition for a good kernel follows immediately from (2). 
\end{enumerate}
\noindent Using the Fej\'er Kernel, we can prove the following theorem. \\

\noindent \textbf{Weierstrass Approximation Theorem}: For every continuous function $f: \mathbb{R} \rightarrow \mathbb{C}$ with period $2\pi$, and for every $\varepsilon>0$, we can find a trigonometric polynomial $P$ such that for all $x$, $\vert f(x) - P(x)\vert < \varepsilon$. 

\begin{proof}
We begin by taking a Ces\'aro mean,
$$\sigma_N(f)(x) = \frac{S_0(f)(x) + \cdots + S_{N-1}(f)(x)}{N} = (f * F_N)(x)$$
\noindent Recall that the partial sums contain powers of $e^{ix}$, therefore, $S_0(f)(x) + \cdots + S_{N-1}(f)(x)$ (that we divide by a constant $N$) is a trigonometric polynomial. Therefore, since $f$ is continuous everywhere and $F_N$ is a family of good kernel, then $(f*F_N) \rightarrow f$ uniformly. 
\end{proof}

\noindent \textbf{Mean-Square Convergence of Fourier Series} \\
\noindent Let $f$ be a continuous function on the circle. Then $\Vert f- S_N(f) \Vert_2 \rightarrow 0$ as $N\rightarrow \infty$.

\begin{proof}
By the Weierstrass Approximation Theorem, for any given $\varepsilon>0$, there exists a trigonometric polynomial $P$ of degree $M$ such that 
$$\vert f(x) - P(x) \vert < \varepsilon \text{ for all } x \in [0,2\pi]$$
\noindent then,
$$\Vert f - P\Vert_2 = \Big( \frac{1}{2\pi} \int^{2\pi}_0 \vert f(x) - P(x)\vert^2 \, dx \Big)^\frac{1}{2} < \varepsilon$$
\noindent If $\vert f(x)-P(x) \vert < \varepsilon$, then we get the average of the function, and in fact, we chose the polynomial $P$ such that it fit this condition, therefore, by the Best Approximation Theorem, then,
$$\Vert f- S_N(f)(x)\Vert_2 < \varepsilon \text{ whenever } N \geq M$$
\end{proof}

\noindent \textbf{Parseval's Identity} \\
\noindent From general theory about orthonormal sequences, we have:
\begin{enumerate}[itemsep=0pt, topsep=0pt, parsep=0pt, partopsep=0pt] 
\item Bessel's Inequality
$$ \sum^\infty_{n=-\infty} \vert \hat{f}(n) \vert^2 \leq \Vert f \Vert_2^2.$$
\item Riemann-Lebesgue Lemma:
$$\hat{f}(n) \rightarrow 0 \text { as } \vert n \vert \rightarrow \infty.$$
In other words, 
$$\int^{2\pi}_0 \sin(nx) \, dx \rightarrow 0 \hspace*{25pt} \text{ and } \hspace*{25pt} 
\int^{2\pi}_0 \cos(nx) \, dx \rightarrow 0$$
\end{enumerate}

\noindent \textbf{Theorem}: Parseval's Identity
\noindent If we assume that $f$ is continuous on the circle, then 
$$\sum^\infty_{n=-\infty} \vert \hat{f}(n) \vert^2 = \Vert f \Vert_2^2$$

\begin{proof}
By the Best Approximation Theorem, we obtain
$$\Vert f \vert^2_2 = \Vert f - S_N(f) \Vert_2^2 + \sum_{\vert n \vert \leq N} \vert \hat{f}(n) \vert^2$$
As $N \rightarrow \infty$, $\Vert f- S_N(f)\Vert_2^2 \rightarrow 0$. So, 
$$\Vert f \Vert_2^2 = \sum_{\vert n \vert \leq N} \vert \hat{f}{(n) \vert^2} \text{ as } N \rightarrow \infty$$
\end{proof}
\end{document}