\documentclass[12pt]{article}
\usepackage{float, amsmath, amssymb, amsthm, algorithm, algorithmic, graphicx, caption, subcaption, mathrsfs, color, cancel, verbatim, cite, authblk, mathtools}
\usepackage{enumitem}

\def\upint{\mathchoice%
    {\mkern13mu\overline{\vphantom{\intop}\mkern7mu}\mkern-20mu}%
    {\mkern7mu\overline{\vphantom{\intop}\mkern7mu}\mkern-14mu}%
    {\mkern7mu\overline{\vphantom{\intop}\mkern7mu}\mkern-14mu}%
    {\mkern7mu\overline{\vphantom{\intop}\mkern7mu}\mkern-14mu}%
  \int}
\def\lowint{\mkern3mu\underline{\vphantom{\intop}\mkern7mu}\mkern-10mu\int}

\let\oldemptyset\emptyset
\let\emptyset\varnothing

\setlength{\oddsidemargin}{0in}
\setlength{\evensidemargin}{\oddsidemargin}
\setlength{\textwidth}{6.5in}
\setlength{\topmargin}{-.25in}
\setlength{\headheight}{0in}
\setlength{\headsep}{0in}
\setlength{\topskip}{0in}
\setlength{\textheight}{9.5in}
\font\bigbf = cmbx10 scaled \magstep1
\font\medbf = cmbx10 scaled \magstephalf
\font\medrm = cmr10 scaled \magstephalf
\font\bigrm = cmr10 scaled \magstep1

\usepackage[english]{babel}
\usepackage[utf8]{inputenc}
\usepackage[colorinlistoftodos]{todonotes}

\title{Harmonic Analysis}
\begin{document}
\noindent \textbf{Lecture 4: Ricardo S\'aenz} \\
\noindent http://fejer.ucol.mx/ricardo/ \\

\noindent \textbf{General Euclidean Space}: $\mathbb{R}^d = \{ (x_1, \dots, x_d) \mid x_i \in \mathbb{R}\}$.  \\

\noindent We will use $x$ for $(x_1,x_2,\dots, x_d)$. For $d=2$, then $x = (x_1, x_2)$.  \\

\noindent Functions of several variables. \\
\noindent Let $u(x): \Omega \rightarrow \mathbb{R}$, where $\Omega \subset \mathbb{R}^d$, where $\Omega$ is open (meaning it does not contain its boundary). \\

\noindent \textbf{Open Ball}: $B_r(x) = \{y \in \mathbb{R}^d : \Vert x-y \Vert < r \}$, where $\Vert \cdot \Vert$ is the normal Euclidean norm. \\

\noindent \textbf{Closed Ball}: $\overline{B}_r(x) = \{y \in \mathbb{R}^d : \Vert x-y \Vert < r \}$, where $\Vert \cdot \Vert$ is the normal Euclidean norm. \\

\noindent \textbf{Generalized Circle}: $S_r(x) = \{y \in \mathbb{R}^d : \Vert x-y \Vert = r\}$.  \\

\noindent \textbf{Notation}: 
$$\frac{\partial u }{\partial x_j} = \partial_{x_j}u = \partial_ju$$
$$\alpha = (\alpha_1, \alpha_2, \dots, \alpha_d), j \in \mathbb{N}$$

\noindent \textbf{The Laplacian in $d$-Dimensions}: \\
\noindent The Laplacian of $u:\Omega \rightarrow \mathbb{R}$ is defined by $\Delta u:= \partial^2_1 u + \partial^2_2 u + \cdots + \partial^2_d u$. We say that $u$ is harmonic in $\Omega$ if $\Delta u(x) = 0$ for all $x \in \Omega$. 
\begin{enumerate}[itemsep=0pt, parsep=0pt, partopsep=0pt, topsep=0pt]
\item Constant functions are harmonic.
\item Linear functions are harmonic.
$$ u(x)= a_1x_2 + a_2x_2 + \cdots a_dx_d + b $$
\item $u(x,y) = xy$ is harmonic in $\mathbb{R}^2$.
\item $u(x,y) = x^2 - y^2$ is harmonic in $\mathbb{R}^2$.
\item $u(x,y) = x^3 - 3xy^2$ is harmonic in $\mathbb{R}^2$. 
\item $u(x,y) = y^3 - 3yx^2$ is harmonic in $\mathbb{R}^2$. 
\end{enumerate}
Whenever you take a holomorphic function, you get harmonic functions. For (5) and (6), if we take a linear combination of the two polynomials, we form a basis for all harmonic functions of degree 3. \\

\noindent \textbf{Mean Value Property}: If $u$ is harmonic in $\Omega$ and $\overline{B}_r(x_0) \subset \Omega$, then
$$\frac{1}{\vert S_r(x_0)\vert} \int_{S_r(x_0)} u(\xi) d\sigma(\xi) = u(x_0)$$

(1)

In $d=2$, then $d\sigma = d\theta$ and
$$\frac{1}{2\pi}\int^{2\pi}_0 u(x_0 + r \cos\theta, t_0 + r\sin \theta)d \theta$$


In $d=3$, then 
$$\frac{1}{4\pi}\int^{2\pi}_0\int^\pi_0 u(\text{spherical coordinates}) \sin\phi \frac{d\phi d \theta}{d\sigma}$$

$$u(x_0) = \frac{1}{\omega_d} \int_{\mathbb{S}} u(x_0 + r \xi) d\sigma(\xi)$$
$$\int_{\mathbb{R}^d} u dx = \int_{\mathbb{S}}\int^\infty_0 u (r \xi)r^{d-1} dr d\sigma$$
where $\omega_d = \frac{2\pi^{d/2}}{\Gamma(d/2)}$. 

For $\alpha>0$, $$\Gamma(\alpha) = \int^\infty_0 t^\alpha e^{-t} \frac{dt}{t}$$
$$\Gamma(1)=1$$
$$\Gamma(\alpha) = (\alpha-1)\Gamma(\alpha-1)$$
$$\Gamma(2) = 1$$ 
$$\Gamma(3) = 2\Gamma(2) = 2$$

$$\Gamma(d+1) = d!$$
The Gamma function generalizes the factorial. 
$$\Gamma(\frac{1}{2}) = \int^\infty_0 t^{1/2} e^{-t} \frac{dt}{t} = \int^\infty_{-\infty} e^{-s^2} ds = \sqrt{\pi}$$

$$\int_{\mathbb{R}^d} e^{-\pi \vert x \vert^2} dx = 1$$

\noindent Some multivariable calculus: Green's Identity. \\
\noindent Let $\Omega$ be a domain with a smooth boundary. For any smooth boundary, we can define a normal vector from each point on the boundary. 
$$ \int_{\partial\Omega} (u\Delta v - v \Delta u) dx = \int (u \partial_{\hat{n}} - v \partial_{\hat{n}}u)  d\sigma $$

\noindent Directional Derivative: $\partial_{\hat{n}} u = \nabla u \cdot \hat{n}$.

\noindent Let $u$ be a harmonic, and let $v=1$. If you integrate a harmonic function over the boundary, you will get zero.
$$ \int_{\partial\Omega} \partial_{\hat{n}}u d\sigma = 0$$

\noindent From a physics perspective, a harmonic function is a case where the heat is flowing out in every direction. 



\end{document}