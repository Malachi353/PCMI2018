\documentclass[12pt]{article}
\usepackage{float, amsmath, amssymb, amsthm, algorithm, algorithmic, graphicx, caption, subcaption, mathrsfs, color, cancel, verbatim, cite, authblk, mathtools}
\usepackage{enumitem}

\def\upint{\mathchoice%
    {\mkern13mu\overline{\vphantom{\intop}\mkern7mu}\mkern-20mu}%
    {\mkern7mu\overline{\vphantom{\intop}\mkern7mu}\mkern-14mu}%
    {\mkern7mu\overline{\vphantom{\intop}\mkern7mu}\mkern-14mu}%
    {\mkern7mu\overline{\vphantom{\intop}\mkern7mu}\mkern-14mu}%
  \int}
\def\lowint{\mkern3mu\underline{\vphantom{\intop}\mkern7mu}\mkern-10mu\int}

\let\oldemptyset\emptyset
\let\emptyset\varnothing

\setlength{\oddsidemargin}{0in}
\setlength{\evensidemargin}{\oddsidemargin}
\setlength{\textwidth}{6.5in}
\setlength{\topmargin}{-.25in}
\setlength{\headheight}{0in}
\setlength{\headsep}{0in}
\setlength{\topskip}{0in}
\setlength{\textheight}{9.5in}
\font\bigbf = cmbx10 scaled \magstep1
\font\medbf = cmbx10 scaled \magstephalf
\font\medrm = cmr10 scaled \magstephalf
\font\bigrm = cmr10 scaled \magstep1

\usepackage[english]{babel}
\usepackage[utf8]{inputenc}
\usepackage[colorinlistoftodos]{todonotes}

\title{Harmonic Analysis}
\begin{document}
\noindent \textbf{Lecture 12: Ricardo S\'aenz} \\
\noindent http://fejer.ucol.mx/ricardo/ \\

\noindent \textbf{The Poisson Kernel for the Upper Half Space} 
$$\mathbb{R}^{d-1}_+ = \{(x,t) : x \in \mathbb{R}^d, t>0\}$$

\noindent The Poisson kernel is given by
$$ P_t(x) = \frac{2t}{\omega_{d+1}(\vert x \vert^2 + t^2)^{\frac{d+1}{2}} }$$
\begin{enumerate}[itemsep=0pt, parsep=0pt, topsep=0pt, partopsep=0pt]
\item The Poisson kernel is $P_t(x) > 0$.
\item $(x,t) \mapsto P_t(x)$ is harmonic in $\mathbb{R}^{d+1}_+$. 
\item $P_t(x)$ is a dilation of $P_1$, so $P_t(x) = \frac{1}{t^d} P_1 (\frac{x}{t})$.
\item For every $t>0$, $\int_{\mathbb{R}^d} P_t(x) \, dx = 1$.
\item (2) For each $\delta> 0$, $\int_{\vert x \vert \geq \delta} P_t(x) \,dx \rightarrow 0$ as $t \rightarrow 0$. 
$$P_t(x) \leq \frac{2t}{\omega_{d+1}\vert x \vert^{d+1}}, \int_{\vert x \vert \geq \delta} P_t(x) \,dx \leq \frac{2t}{\omega_{d+1} \delta} \rightarrow 0$$

\noindent We define the Poisson integral of $f \in L'(\mathbb{R})$ (this just means that the integral is finite). There needs to be some decay in order for the integral to be finite. So bounded functions are not necessarily finite integrable.

\noindent Let $$u(x,t) = P_t * f(x) = \int_{\mathbb{R}^d} P_y(t) f(x-y) \, dy$$

\item Let $x \mapsto u(x,t)$ is integrable on $\mathbb{R}^d$. 
$$ \int \vert P_t * f(x) \vert \, dx \leq \iint P_t(y) \vert f(x-y) \,dy \,dx \leq  \int P_t(y) \int \vert f(x-y)\vert \,dx \,dy = \int \vert f \vert'$$


If you have any function $\int_{\mathbb{R}^d} \frac{1}{t^d} F(\frac{x}{t} \, dx = \int_{\mathbb{R}^d} F(x) dx$. 

\noindent So $\{P_t(x)\}_{t>0}$ is a collection of good kernels.
\item If $f \in C_c(\mathbb{R}^d)$, then $\lim_{t \rightarrow 0} u (x,t) = f(x)$ uniformly in $x$. 

\item  If $f \in L^1(\mathbb{R}^d)$, then $\int \vert u(x,t) - f(x) \vert \, dx \rightarrow 0$ as $t \rightarrow 0$. We estimate $f$ with a continuous function, so let
$$g \in C_c, \Vert f-g \Vert_1 < \varepsilon$$
$$ \Vert P_t(f) - f \Vert_1 \leq \Vert P_t(f-g)\Vert_1 + \Vert P_t(g) -g \Vert_1 + \Vert g-f \Vert_1$$

We want to estimate the whole integral. When we have $$\int \Vert P_tg(x) - g(x) \vert \,dx \leq \iint P_t(y) \vert g(x-y) - g(x)\vert \,dy \,dx$$
$$\leq \int P_t(y) \int \vert g(x-y) - g(x) \vert \, dx \, dy$$
We split the integral into two different integrals with different bounds. For some, the domain of the first will be $<N$ and the second will be $\geq N$. We choose $N$ in $\int \vert f \vert < \infty$. \\

\noindent There is something that is not like before in the other context. Let $f \in L^1(\mathbb{R}^d)$. $f$ is not necessarily continuous, it could be discontinuous every. Is
$$\lim_{t\rightarrow 0 }u(x,t) = f(x) \text{ at each } x ?$$
It is not uniformly convergent. \\

Almost every $x$, if the set where it does not work has measure 0. if it can be covered by arbitrarily small covers. For any $\varepsilon$, we can cover the set of all problem points with balls, and the volume of the balls will be less than $\varepsilon$. 
\end{enumerate}

(3) 

It is possible for a sequence of function $\int \vert f_n - f \vert \rightarrow$ but $f_n(x) \not\rightarrow f(x)$ for all $x$ .

(4)

Let $\alpha>0$, and $E_{\alpha} = \{ \limsup_t\rightarrow 0 \vert u(x,t) - f(x) \vert > \alpha\}$

It is the upper bound on all the limits, the limit infimum is the lower bound on all the limits. If limit supremum and limit infimum is equal. 

We want to prove that $E_\alpha$ has measure 0. We have to approximate using a continuous function. 

$$P_t(x) * f(x) - f(x) = P_t*(f-g) + P_t(g(x)-g(x) + g(x) - f(x)$$
for $g \in C$ such that $\Vert g-f \Vert_1 < \varepsilon$. 

$g(x)-f(x)$ is not a problem becuase if you take the measure of the size
$$ \vert \{ x : \vert f(x) - g(x) \vert > \alpha \} \vert = \int_{\vert f-g\vert > \alpha} 1 \,dx \leq \int \frac{\vert f-g\vert }{\alpha} \, dx
 = \frac{1}{\alpha} \Vert f-g \Vert_1 < \frac{\varepsilon}{\alpha} $$
 
$$\int P_t(y) (f(x-y)-g(x-y)) \,dx = ?$$

Hardy-Littlewood maximal function. Let $f \in L^1(\mathbb{R}^d)$, we will consider 
$$Mf(x) = \sup_{r>0}\frac{1}{\vert B_r(x)\vert}\int_{B_r(x)} \vert f(y)\vert \,dy $$

We consider all possible balls around $x$, larger balls make the integral go to zero because you divide by its volume. The problem is when $r$ is small. If we can control the maximal function, then we can control the original integral we are trying to estimate above. \\ 

\noindent \textbf{Theorem}: There exists a constant $A > 0$ such that for every $f \in L^1(\mathbb{R}^d)$
$$\vert u(x,t) \vert \leq A Mf(x) \text{ for all } t > 0$$
(5)

\noindent The idea is to take a Poisson integral, and split the integral into parts. We are going to split the integral into
$$u(c,y) = \int P_t(y)f(x-y) \,dy$$
$$ = \int_{\vert y \vert \leq t} + \sum_{j=1}^\infty \int_{2t < \vert y \vert \leq 2^jt}$$

Each one of the integrals can be compared to an average of $f$. The first is given by 
$$\int_{\vert y \vert \leq t} \frac{2t}{\omega_{d+1}(\vert y\vert^2 + \vert x \vert^2)^{(d+1)/2}} \vert f(y) \vert \,dy \leq \frac{2}{\omega_{d+1}} \frac{1}{t^d} \int_{\vert y \vert \leq t} \vert f(x-y) \vert \,dt = \frac{2\omega_d}{\omega_{d+1}} \frac{1}{\vert B_t(x)\vert } \int_{\vert B_t(x)\vert } \vert f \vert \leq A Mf(x)$$
$$ \int_{2^{j-1}t < \vert y \vert \leq 2^jt} \frac{2t}{\omega_{d+1}(\vert y\vert^2 + \vert x \vert^2)^{(d+1)/2}} \vert f(x-y) \vert \,dy $$ 
$$\leq \int_{2^{j-1}t < \vert y \vert \leq 2^jt} \frac{2t}{\omega_{d+1}}(\vert y\vert^{d+1} \vert f(x-y) \vert \,dy $$
$$\leq A \frac{t}{(2^{j-i}t)^{d+1}} \int_{2^{j-1}t < \vert y \vert \leq 2^jt} \vert f(x-y) \vert \,dy$$
$$\leq \frac{A}{2^j \vert B_{2^it}(x)\vert} \int_{B_2^it} \vert f \vert \leq \frac{A}{2^j} Mf(x)$$
Therefore, 
$$\vert u(x,t) \vert \leq \sum^\infty_{j=0} \frac{A}{2^j} Mf(x) = 2 A Mf(x)$$

\end{document}