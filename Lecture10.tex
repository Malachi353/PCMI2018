\documentclass[12pt]{article}
\usepackage{float, amsmath, amssymb, amsthm, algorithm, algorithmic, graphicx, caption, subcaption, mathrsfs, color, cancel, verbatim, cite, authblk, mathtools}
\usepackage{enumitem}

\def\upint{\mathchoice%
    {\mkern13mu\overline{\vphantom{\intop}\mkern7mu}\mkern-20mu}%
    {\mkern7mu\overline{\vphantom{\intop}\mkern7mu}\mkern-14mu}%
    {\mkern7mu\overline{\vphantom{\intop}\mkern7mu}\mkern-14mu}%
    {\mkern7mu\overline{\vphantom{\intop}\mkern7mu}\mkern-14mu}%
  \int}
\def\lowint{\mkern3mu\underline{\vphantom{\intop}\mkern7mu}\mkern-10mu\int}

\let\oldemptyset\emptyset
\let\emptyset\varnothing

\setlength{\oddsidemargin}{0in}
\setlength{\evensidemargin}{\oddsidemargin}
\setlength{\textwidth}{6.5in}
\setlength{\topmargin}{-.25in}
\setlength{\headheight}{0in}
\setlength{\headsep}{0in}
\setlength{\topskip}{0in}
\setlength{\textheight}{9.5in}
\font\bigbf = cmbx10 scaled \magstep1
\font\medbf = cmbx10 scaled \magstephalf
\font\medrm = cmr10 scaled \magstephalf
\font\bigrm = cmr10 scaled \magstep1

\usepackage[english]{babel}
\usepackage[utf8]{inputenc}
\usepackage[colorinlistoftodos]{todonotes}

\title{Harmonic Analysis}
\begin{document}
\noindent \textbf{Lecture 8: Ricardo S\'aenz} \\
\noindent http://fejer.ucol.mx/ricardo/ \\

\noindent \textbf{Spherical Harmonics} \\
\noindent We are interested in developing the Fundamental Harmonics on the Sphere (for any dimension). Let $\mathcal{P}_k$ be the set of homogeneous polynomials of degree $k$ with $d$ variables. \\

\textbf{Example}: Let $d=3$, then
\begin{align*} P_4 &= \{x_1^4,x_1^2x_2^2, x_1^4 + 2x_2^2x_2^2 - x_3^4, \dots \} \\
&= \text{span}\{x_1^{\alpha_1}x_2^{\alpha_2}x_3^{\alpha_3} = 4\}
\end{align*}

\begin{enumerate}[itemsep=0pt, topsep=0pt, parsep=0pt, partopsep=0pt]
\item The set $\mathcal{P}_k$ forms a vector space with dimension
$$\dim(\mathcal{P}_k) = \binom{k+d-1}{k},$$
where $d$ is the number of variables. To see this, find how many solutions to the equation $\alpha_1 + \cdots + \alpha_d = k$ (or how many ways to buy $k$ bagels with $d$ choices). \\ 
\end{enumerate}

\noindent Let $p \in \mathcal{P}_k$, then we have the following identities:\\
\begin{enumerate}[itemsep=0pt, topsep=0pt, parsep=0pt, partopsep=0pt]
\addtocounter{enumi}{1}
\item For $t \in \mathbb{R}$, then $p(t\textbf{x}) = t^kp(\textbf{x})$. \\
\item (Euler's Identity) The polynomial $p \in \mathcal{P}_k$ if and only if $\sum x_j \partial_j p = kp$ (in fact, this property holds for all homogeneous differentiable functions, but for our purpose, we focus on all homogeneous polynomials). \\
\end{enumerate}

\noindent Let $\mathcal{H}_k = \{p \in \mathcal{P}_k : \Delta p=0\}$. Using this space of polynomials, we can find infinitely many harmonics by taking linear combinations of its basis elements. Therefore, it is of much interest to find this basis. \\

\begin{enumerate}[itemsep=0pt, topsep=0pt, parsep=0pt, partopsep=0pt]
\addtocounter{enumi}{3}
\item For all $k \in \mathbb{N}$, $\mathcal{P}_k = \mathcal{H}_k \oplus \vert x \vert^2 \mathcal{P}_{k-2}$, where $\oplus$ is a direct sum of vector spaces. \\
\end{enumerate}

\noindent  This means, for all $p \in \mathcal{P}_k$, there exists $h \in \mathcal{H}_k$ and $f \in \mathcal{P}_{k-2}$ such that $p = h + f$. If $\vert x \vert^2 \mathcal{P}_{k-2}$ and $\mathcal{H}_k$ share no basis elements ($\vert x \vert^2 \mathcal{P}_{k-2} \cap \mathcal{H}_k = \{0\}$), then $h+f$ is unique and 
$\dim(\mathcal{P}_k) = \dim(\mathcal{H}_k) + \dim(\vert x \vert^2 \mathcal{P}_{k-2})$. Using (4), we can show the dimension of $\mathcal{H}_k$. \\

\begin{enumerate}[itemsep=0pt, topsep=0pt, parsep=0pt, partopsep=0pt]
\addtocounter{enumi}{4}
\item For all $k \in \mathbb{N}$, 
$$\dim(\mathcal{H}_k) = \binom{k+d-1}{k} - \binom{k+d-3}{k-2} = \frac{(2k+d-2)(k+d-3)!}{k!(d-2)!}$$
\end{enumerate}

\end{document}